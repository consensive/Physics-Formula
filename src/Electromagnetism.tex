\documentclass[10pt]{jarticle}

\renewcommand{\contentsname}{電磁気}
\usepackage{kousiki}


\begin{document}
%%%%%%%%%%%%%%%%%%%%%%%%%%%%%
%Foreword Page
\addtocounter{page}{-1}
\thispagestyle{empty}

まえがき\\

{\scriptsize
『初歩から学ぶ基礎物理学 電磁気・原子』(第日本図書)の電磁気分野で現れる法則・公式をまとめました.\\

 演習は単なる算数ではなく思考の実体験の場です.
意味記憶だけではなく,エピソード記憶として法則・公式を自身の思考に取り入れてもらえることを願っています.\\

 習得してから公式集を振り返ると,物理教師がよく言う「公式は暗記するものではない.理解するものだ」という台詞の意味を実感してもらえるはずです.\\



\hfill
釧路高専(物理)

\hfill
松崎俊明

\vfill
\hfill
(2016-12-15)
}



%%%%%%%%%%%%%%%%%%%%%%%%%%%%%
%TOC Page
\addtocounter{page}{-1}
\thispagestyle{empty}
\tableofcontents



\newpage
%%%%%%%%%%%%%%%%%%%%%%%%%%%%%
%Title Page
\addtocounter{page}{-1}
\thispagestyle{empty}
\section{電場}

\begin{enumerate}
\small
\itemsep-4mm
\item 真空のクーロン定数 \\
\item クーロンの法則 \\
\item 電気素量 \\
\item 電場と力 \\
\item 電場と力線の密度 \\
\item ガウスの法則 \\
\item 点電荷が作る電場 \\
\item 無限平面が作る電場 \\
\item 無限直線が作る電場 \\
\end{enumerate}
%%%%%%%%%%%%%%%%%%%%%%%%%%%%%


\newpage
%%%%%%%%%%%%%%%%%%%%%%%%%%%%%
%Title Page
\addtocounter{page}{-1}
\thispagestyle{empty}

\vspace*{-10mm}
\begin{enumerate}
\setcounter{enumi}{9}
\small
\itemsep-4mm
\item 電位の定義 \\
\item 一様電場が作る電位 \\
\item 点電荷が作る電位 \\
\item 電位と電場 \\
\item 電気容量 \\
\item 平行板コンデンサー \\
\item 誘電率 \\
\item 合成容量(並列) \\
\item 合成容量(直列) \\
\item コンデンサーのエネルギー \\
\end{enumerate}




\newpage
%%%%%%%%%%%%%%%%%%%%%%%%%%%%%
\[
	k_0 = 9.0 \! \times \! 10^9  \mathrm{(N \! \cdot \! m^2/C^2)}
\]


\vskip3mm
【真空のクーロン定数】\\
\hfill {\footnotesize (p. 11)}





\newpage
%%%%%%%%%%%%%%%%%%%%%%%%%%%%%
\[
F = k \frac{q_1 q_2}{r^2}
\]


\vskip3mm
【クーロンの法則】{\footnotesize (p. 11)}

\begin{tabular}{ccl}
$F$	&[N]	&:力 \\
$q_i$	&[C]	&:電荷 \\
$r$	&[m]	&:距離
\end{tabular}




\newpage
%%%%%%%%%%%%%%%%%%%%%%%%%%%%%
\[
e = 1.60 \! \times \! 10^{-19} \mathrm{(C)}
\]


\vskip3mm
【電気素量】{\footnotesize (p. 14)}




\newpage
%%%%%%%%%%%%%%%%%%%%%%%%%%%%%
\[
\vec{F} = q \vec{E}
\]


\vskip3mm
【電場と力】{\footnotesize (p. 18, 21)}

\begin{tabular}{ccl}
$\vec{F}$	&[N]	&:力 \\
$q$	&[C]	&:電荷 \\
$\vec{E}$	&[N/C]	&:電場
\end{tabular}





\newpage
%%%%%%%%%%%%%%%%%%%%%%%%%%%%%
\[
  E = \frac{N}{S}
\]


\vskip3mm
【電場と力線の密度】{\footnotesize (p. 23)}

\begin{tabular}{ccl}
$E$	&[N/C]	&:電場の強さ \\
$N$	&[本]	&:電気力線の本数 \\
$S$	&[m$^2$]	&:力線が貫く面積
\end{tabular}



\newpage
%%%%%%%%%%%%%%%%%%%%%%%%%%%%%
\[
N = 4\pi k Q
\]


\vskip3mm
【ガウスの法則】{\footnotesize (p. 25)}

\begin{tabular}{ccl}
$N$	&[本]	&:電気力線の本数 \\
$Q$	&[C]	&:電荷
\end{tabular}






\newpage
%%%%%%%%%%%%%%%%%%%%%%%%%%%%%
\[
E = k \frac{Q}{\; r^2 \;}
\]
\vskip3mm
【点電荷が作る電場】{\footnotesize (p. 19)}

\begin{tabular}{ccl}
$E$	&[N/C]	&:電場 \\
$Q$	&[C]	&:電荷 \\
$r$	&[m]	&:距離
\end{tabular}





\newpage
%%%%%%%%%%%%%%%%%%%%%%%%%%%%%
\[
E = 2 \pi k \sigma
\]
\vskip3mm
【無限平面が作る電場】{\footnotesize (p. 26)}

\begin{tabular}{ccl}
$E$	&[N/C]	&:電場 \\
$\sigma$	&[C/m$^2$]	&:面密度
\end{tabular}




\newpage
%%%%%%%%%%%%%%%%%%%%%%%%%%%%%
\[
E = \frac{2k \sigma}{r}
\]
\vskip3mm
【無限直線が作る電場】{\footnotesize (p. 27)}

\begin{tabular}{ccl}
$E$	&[N/C]	&:電場 \\
$\sigma$	&[C/m]	&:線密度\\
$r$	&[m]	&:距離
\end{tabular}




\newpage
%%%%%%%%%%%%%%%%%%%%%%%%%%%%%
\[
V = \frac{W}{Q} = \int \vec{E} \cdot d\vec{s}
\]


\vskip3mm
【電位の定義】{\footnotesize (p. 31)}

\begin{tabular}{ccl}
$V$	&[V]	&:電位 \\
$W$	&[J]	&:{\small 静電気力がする仕事} \\
$Q$	&[C]	&:電荷\\
$E$	&[V/m]	&:電場の強さ\\
$ds$	&[m]	&:微小な移動距離\\
\end{tabular}




\newpage
%%%%%%%%%%%%%%%%%%%%%%%%%%%%%
\[
V = E d
\]


\vskip3mm
【一様電場が作る電位】{\footnotesize (p. 32)}

\begin{tabular}{ccl}
$V$	&[V]	&:電位 \\
$E$	&[N/C=V/m]	&:電場 \\
$d$	&[m]	&:距離
\end{tabular}





\newpage
%%%%%%%%%%%%%%%%%%%%%%%%%%%%%
\[
V = k \frac{Q}{\; r \;}
\]


\vskip3mm
【点電荷が作る電位】{\footnotesize (p. 38)}

\begin{tabular}{ccl}
$V$	&[V]	&:電位 \\
$Q$	&[C]	&:電荷 \\
$r$	&[m]	&:距離
\end{tabular}



\newpage
%%%%%%%%%%%%%%%%%%%%%%%%%%%%%
\[
\vec{E} = - \vec{\nabla} V
\]


\vskip3mm
【電位と電場】{\footnotesize (p. 43)}

\begin{tabular}{ccl}
$E$	&[N/C=V/m]	&:電場 \\
$V$	&[V]	&:電位
\end{tabular}





\newpage
%%%%%%%%%%%%%%%%%%%%%%%%%%%%%
\[
Q = C V
\]


\vskip3mm
【電気容量】{\footnotesize (p. 43)}

\begin{tabular}{ccl}
$Q$	&[C]	&:電荷 \\
$C$	&[F]	&:静電容量 \\
$V$	&[V]	&:電位
\end{tabular}





\newpage
%%%%%%%%%%%%%%%%%%%%%%%%%%%%%
\[
C = \frac{1}{4\pi k} \frac{S}{\, d \,} = \varepsilon \frac{S}{\, d \,}
\]


\vskip3mm
【平行板コンデンサー】{\footnotesize (p. 51)}

\begin{tabular}{ccl}
$C$	&[F]	&:静電容量 \\
$S$	&[m$^2$]	&:面積 \\
$d$	&[m]	&:距離 \\
$\varepsilon$	&[F/m]	&:誘電率
\end{tabular}




\newpage
%%%%%%%%%%%%%%%%%%%%%%%%%%%%%
\[
\varepsilon = \varepsilon_r \varepsilon_0
\]


\vskip3mm
【誘電率】{\footnotesize (p. 53)}

\begin{tabular}{ccl}
$\varepsilon$	&[F/m]	&:静電率 \\
$\varepsilon_r$	&	&:比誘電率\\
$\varepsilon_0$	&[F/m]	&:真空の誘電率
\end{tabular}






\newpage
%%%%%%%%%%%%%%%%%%%%%%%%%%%%%
\[
C = C_1 + \cdots + C_n
\]


\vskip3mm
【合成容量(並列)】{\footnotesize (p. 55)}

\begin{tabular}{ccl}
$C$	&[F]	&:合成容量 \\
$C_i$	&[F]	&:静電容量 \\
\end{tabular}




\newpage
%%%%%%%%%%%%%%%%%%%%%%%%%%%%%
\[
\frac{1}{C} = \frac{1}{C_1} + \cdots + \frac{1}{C_n}
\]


\vskip3mm
【合成容量(直列)】{\footnotesize (p. 57)}

\begin{tabular}{ccl}
$C$	&[F]	&:合成容量 \\
$C_i$	&[F]	&:静電容量 \\
\end{tabular}



\newpage
%%%%%%%%%%%%%%%%%%%%%%%%%%%%%
\[
U = \frac{1}{\, 2 \,} C V^2
\]


\vskip3mm
【コンデンサーのエネルギー】\\
\hfill {\footnotesize (p. 60)}

\begin{tabular}{ccl}
$U$	&[J]	&:エネルギー \\
$C$	&[F]	&:静電容量 \\
$V$	&[V]	&:電位
\end{tabular}






\newpage
%%%%%%%%%%%%%%%%%%%%%%%%%%%%%
%Title Page
\addtocounter{page}{-1}
\thispagestyle{empty}
\section{電流}

\begin{enumerate}
\setcounter{enumi}{\thepage}
\small
\itemsep-4mm
\item 電荷と電流 \\
\item 自由電子と電流 \\
\item オームの法則 \\
\item 抵抗率 \\
\item 温度係数 \\
\item 電力 \\
\item 電流密度 \\
\item 合成抵抗(直列) \\
\item 合成抵抗(並列) \\
\end{enumerate}
%%%%%%%%%%%%%%%%%%%%%%%%%%%%%





\newpage
%%%%%%%%%%%%%%%%%%%%%%%%%%%%%
\[
I = \frac{q}{\; t \;}
\]


\vskip3mm
【電荷と電流】{\footnotesize (p. 62)}


\begin{tabular}{ccl}
$I$	&[A]	&:電流 \\
$q$	&[C]	&:移動した電荷\\
$t$	&[s]	&:経過時間
\end{tabular}




\newpage
%%%%%%%%%%%%%%%%%%%%%%%%%%%%%
\[
I = en \bar{v} S
\]


\vskip3mm
【自由電子と電流】{\footnotesize (p. 64)}


\begin{tabular}{ccl}
$I$	&[A]	&:電流 \\
$e$	&[C]	&:電気素量\\
$n$	&[個/m$^3$]	&:自由電子密度\\
$\bar{v}$	&[m/s]	&:平均速度\\
$S$	&[m$^2$]	&:断面積
\end{tabular}






\newpage
%%%%%%%%%%%%%%%%%%%%%%%%%%%%%
\[
V = R I
\]


\vskip3mm
【オームの法則】{\footnotesize (p. 65)}

\begin{tabular}{ccl}
$V$	& [V]	&:電圧\\
$R$	&[$\Omega$]	&:抵抗\\
$I$	&[A]	&:電流
\end{tabular}



\newpage
%%%%%%%%%%%%%%%%%%%%%%%%%%%%%
\[
R = \rho \frac{l}{\; S \;}
\]


\vskip3mm
【抵抗率】{\footnotesize (p. 67)}

\begin{tabular}{ccl}
$R$	&[$\Omega$]	&:抵抗\\
$\rho$	&[$\Omega \cdot $m]	&:抵抗率\\
$l$	&[m]	&:長さ\\
$S$	&[m$^2$]	&:断面積
\end{tabular}





\newpage
%%%%%%%%%%%%%%%%%%%%%%%%%%%%%
\[
\rho = \rho_0 (1 + \alpha t)
\]


\vskip3mm
【温度係数】{\footnotesize (p. 68)}

\begin{tabular}{ccl}
$\rho$	&[$\Omega \cdot $m]	&:抵抗率\\
$\rho_0$	&[$\Omega \cdot $m]	&:0℃での抵抗率\\
$\alpha$	&[1/K]	&:温度係数\\
$t$	&[℃]	&:温度
\end{tabular}





\newpage
%%%%%%%%%%%%%%%%%%%%%%%%%%%%%
\[
P = \frac{W}{t} = I V
\]


\vskip3mm
【電力】{\footnotesize (p. 70)}

\begin{tabular}{ccl}
$P$	&[W]	&:電力\\
$W$	& [J]	&:電力量\\
$t$	& [s]	&:時間\\
$V$	& [V]	&:電圧\\
$I$	&[A]	&:電流
\end{tabular}


\newpage
%%%%%%%%%%%%%%%%%%%%%%%%%%%%%
\[
\vec{i} = \frac{\vec{I}}{\; S \;} = \sigma \vec{E}
\]


\vskip3mm
【電流密度】{\footnotesize (p. 71)}

\begin{tabular}{ccl}
$\vec{i}$	&[A/m$^2$]	&:電流密度\\
$\vec{I}$	&[A]	&:電流\\
$S$	&[m$^2$]	&:断面積\\
$\sigma$	& [1/$\Omega \cdot$m]	&:導電率\\
$\vec{E}$	& [V/m]	&:電場
\end{tabular}





\newpage
%%%%%%%%%%%%%%%%%%%%%%%%%%%%%
\[
R = R_1 + \cdots + R_n
\]


\vskip3mm
【合成抵抗(直列)】{\footnotesize (p. 78)}

\begin{tabular}{ccl}
$R$	&[$\Omega$]	&:合成抵抗 \\
$R_i$	&[$\Omega$]	&:抵抗 \\
\end{tabular}





\newpage
%%%%%%%%%%%%%%%%%%%%%%%%%%%%%
\[
\frac{1}{R \;} = \frac{1}{R_1} + \cdots + \frac{1}{R_n}
\]


\vskip3mm
【合成抵抗(並列)】{\footnotesize (p. 80)}

\begin{tabular}{ccl}
$R$	&[$\Omega$]	&:合成抵抗 \\
$R_i$	&[$\Omega$]	&:抵抗 \\
\end{tabular}








\newpage
%%%%%%%%%%%%%%%%%%%%%%%%%%%%%
%Title Page
\addtocounter{page}{-1}
\thispagestyle{empty}
\section{電流と磁場}

\begin{enumerate}
\setcounter{enumi}{\thepage}
\small
\itemsep-4mm
\item 磁気のクーロンの法則 \\
\item 自由電子と電流 \\
\item オームの法則 \\
\item 抵抗率 \\
\item 温度係数 \\
\item 電力 \\
\item 電流密度 \\
\item 合成抵抗(直列) \\
\item 合成抵抗(並列) \\
\end{enumerate}
%%%%%%%%%%%%%%%%%%%%%%%%%%%%%



\newpage
%%%%%%%%%%%%%%%%%%%%%%%%%%%%%
\[
F = k_m \frac{m_1 m_2}{r^2}
\]


\vskip3mm
【磁気のクーロンの法則】\\
\hfill {\footnotesize (p. 11)}

\begin{tabular}{ccl}
$F$	&[N]	&:力 \\
$m_i$	&[Wb]	&:磁気量 \\
$r$	&[m]	&:距離
\end{tabular}








\newpage
%%%%%%%%%%%%%%%%%%%%%%%%%%%%%
\[
\vec{F} = m \vec{H}
\]


\vskip3mm
【磁場】{\footnotesize (p. 89)}

\begin{tabular}{ccl}
$\vec{F}$	&[N]	&:力 \\
$m$	&[Wb]	&:磁気量 \\
$\vec{H}$	&[N/Wb]	&:磁場
\end{tabular}






\newpage
%%%%%%%%%%%%%%%%%%%%%%%%%%%%%
\[
H = k_m \frac{m}{\; r^2 \;}
\]


\vskip3mm
【磁極が作る磁場】{\footnotesize (p. 91)}

\begin{tabular}{ccl}
$H$	&[N/Wb]	&:磁場 \\
$m$	&[Wb]	&:磁気量 \\
$r$	&[m]	&:距離
\end{tabular}





\newpage
%%%%%%%%%%%%%%%%%%%%%%%%%%%%%
\[
H = \frac{I}{2 \pi r}
\]


\vskip3mm
【直線電流が作る磁場】{\footnotesize (p. 94)}

\begin{tabular}{ccl}
$H$	&[A/m]	&:磁場 \\
$I$	&[A]	&:電流 \\
$r$	&[m]	&:電流からの距離
\end{tabular}




\newpage
%%%%%%%%%%%%%%%%%%%%%%%%%%%%%
\[
H = N \frac{I}{\, 2 r \,}
\]


\vskip3mm
【円電流が中心に作る磁場】\\
\hfill {\footnotesize (p. 95)}

\begin{tabular}{ccl}
$H$	&[A/m]	&:磁場 \\
$N$	&[回]	&:巻き数 \\
$I$	&[A]	&:電流 \\
$r$	&[m]	&:円電流の半径
\end{tabular}





\newpage
%%%%%%%%%%%%%%%%%%%%%%%%%%%%%
\[
H = n I
\]


\vskip3mm
【ソレノイドが作る磁場】\\
\hfill {\footnotesize (p. 96)}

\begin{tabular}{ccl}
$H$	&[A/m]	&:磁場 \\
$n$	&[回/m]	&:単位長さの巻数\\
$I$	&[A]	&:電流 \\
\end{tabular}





\newpage
%%%%%%%%%%%%%%%%%%%%%%%%%%%%%
\[
\mathit{\Delta} H = \frac{I \mathit{\Delta}l \sin\theta}{4\pi r^2}
\]


\vskip3mm
【ビオ・サバールの法則】\\
\hfill {\footnotesize (p. 97)}

\begin{tabular}{ccl}
$\mathit{\Delta} H$	&[A/m]	&:微小磁場 \\
$I$	&[A]	&:電流\\
$\mathit{\Delta} l$	&[m]	&:微小区間\\
$\theta$	&[rad]	&:角度\\
$r$	&[m]	&:距離
\end{tabular}




\newpage
%%%%%%%%%%%%%%%%%%%%%%%%%%%%%
\[
\sum_i I_i = \oint \vec{H} \cdot d \vec{s}
\]


\vskip3mm
【アンペールの法則】\hfill {\footnotesize (p. 100)}

\begin{tabular}{ccl}
$I_i$	&[A]	&:電流\\
$\vec{H}$	&[A/m]	&:磁場\\
$d \vec{s}$	&[m]	&:微小変位
\end{tabular}




\newpage
%%%%%%%%%%%%%%%%%%%%%%%%%%%%%
\[
\vec{B} = \mu \vec{H}
\]


\vskip3mm
【磁束密度と磁場】{\footnotesize (p. 105)}

\begin{tabular}{ccl}
$\vec{B}$	&[T]	&:磁束密度\\
$\mu$	&[N/A$^2$]	&:透磁率\\
$\vec{H}$	&[A/m]	&:磁場
\end{tabular}




\newpage
%%%%%%%%%%%%%%%%%%%%%%%%%%%%%
\[
\mu = \mu_r \mu_0
\]


\vskip3mm
【透磁率】{\footnotesize (p. 105)}

\begin{tabular}{ccl}
$\mu$	&[N/A$^2$]	&:透磁率\\
$\mu_r$	&	&:比透磁率\\
$\mu_0$	&[N/A$^2$]	&:真空の透磁率
\end{tabular}




\newpage
%%%%%%%%%%%%%%%%%%%%%%%%%%%%%
\[
F = I B l
\]


\vskip3mm
【電流が受ける力】{\footnotesize (p. 106)}

\begin{tabular}{ccl}
$F$	&[N]	&:力 \\
$I$	&[A]	&:電流\\
$B$	&[T]	&:磁束密度\\
$l$	&[m]	&:導線の長さ\\
\end{tabular}




\newpage
%%%%%%%%%%%%%%%%%%%%%%%%%%%%%
\[
\Phi = B S
\]


\vskip3mm
【磁束と磁束密度】{\footnotesize (p. 106)}

\begin{tabular}{ccl}
$\Phi$	&[Wb]	&:磁束\\
$B$	&[T]	&:磁束密度\\
$S$	&[m$^2$]	&:面積
\end{tabular}





\newpage
%%%%%%%%%%%%%%%%%%%%%%%%%%%%%
\[
f = q v B \sin \theta
\]


\vskip3mm
【ローレンツ力】{\footnotesize (p. 109)}

\begin{tabular}{ccl}
$f$	&[N]	&:力 \\
$q$	&[C]	&:電荷\\
$v$	&[m/s]	&:速さ\\
$B$	&[T]	&:磁束密度\\
$\theta$	&[rad]	&:角度
\end{tabular}





\end{document}
