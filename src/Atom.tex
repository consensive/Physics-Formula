\documentclass[10pt]{jarticle}

\renewcommand{\contentsname}{原子}
\usepackage{kousiki}

\begin{document}
%%%%%%%%%%%%%%%%%%%%%%%%%%%%%
%Foreword Page
\addtocounter{page}{-1}
\thispagestyle{empty}

まえがき\\

{\scriptsize
『初歩から学ぶ基礎物理学 電磁気・原子』(第日本図書)の原子分野に現れる法則・公式をまとめました.\\

 演習は単なる算数ではなく思考の実体験の場です.
意味記憶だけではなく,エピソード記憶として法則・公式を自身の思考に取り入れてもらえることを願っています.\\

 習得してから公式集を振り返ると,物理教師がよく言う「公式は暗記するものではない.理解するものだ」という台詞の意味を実感してもらえるはずです.




\vfill
\hfill 釧路高専(物理) 松崎俊明\\
\hfill https://consensive.github.io\\

\vskip-3mm \hfill ver.2017-04-03\\
}



%%%%%%%%%%%%%%%%%%%%%%%%%%%%%
%TOC Page
%\addtocounter{page}{-1}
%\thispagestyle{empty}
%\tableofcontents



\newpage
%%%%%%%%%%%%%%%%%%%%%%%%%%%%%
%Title Page
\addtocounter{page}{-1}
\thispagestyle{empty}
\section{原子}

\begin{enumerate}
\small
\itemsep-4mm
\item 電子の比電荷 \\
\item 電子の電荷 \\
\item 電子の質量 \\
\item 電子ボルト \\
\item プランク定数 \\
\item 光子のエネルギー \\
\item 光子の運動量 \\
\item リュードベリの公式 \\
\item 量子条件 \\
\end{enumerate}
%%%%%%%%%%%%%%%%%%%%%%%%%%%%%


\newpage
%%%%%%%%%%%%%%%%%%%%%%%%%%%%%
%Title Page
\addtocounter{page}{-1}
\thispagestyle{empty}

\vspace*{-10mm}
\begin{enumerate}
\setcounter{enumi}{9}
\small
\itemsep-4mm
\item ブラッグの法則 \\
\item ドブロイ波長 \\
\item 半減期 \\
\item アインシュタインの関係式
\end{enumerate}




\newpage
%%%%%%%%%%%%%%%%%%%%%%%%%%%%%
\[
  \frac{e}{m} = 1.7588 \! \times \! 10^{11}  \mathrm{C/kg}
\]


\vskip3mm
\noindent
【電子の比電荷】{\footnotesize (p. 171)}


\newpage
%%%%%%%%%%%%%%%%%%%%%%%%%%%%%
\[
  e = 1.6022 \! \times \! 10^{-19}  \mathrm{C}
\]


\vskip3mm
\noindent
【電子の電荷】{\footnotesize (p. 174)}




\newpage
%%%%%%%%%%%%%%%%%%%%%%%%%%%%%
\[
  m = 9.109 \! \times \! 10^{-31}  \mathrm{kg}
\]

\vskip3mm
\noindent
【電子の質量】{\footnotesize (p. 174)}




\newpage
%%%%%%%%%%%%%%%%%%%%%%%%%%%%%
\[
  1\mathrm{eV} = 1.6022\! \times\! 10^{-19} \mathrm{J}
\]

\vskip3mm
\noindent
【電子ボルト】{\footnotesize (p. 175)}

定義をしっかりと覚えておけば自明な変換式




\newpage
%%%%%%%%%%%%%%%%%%%%%%%%%%%%%
\[
  h = 6.626 \! \times \! 10^{-34}  \mathrm{J\! \cdot\! s}
\]

\vskip3mm
\noindent
【プランク定数】{\footnotesize (p. 183)}





\newpage
%%%%%%%%%%%%%%%%%%%%%%%%%%%%%
\[
E = h \nu
\]


\vskip3mm
\noindent
【光子のエネルギー】{\footnotesize (p. 185)}

\begin{tabular}{ccl}
$E$	&[J]	&:光子のエネルギー \\
  $h$	&{[J$\cdot$s]}	&:プランク定数 \\
$\nu$	&[Hz]	&:光の振動数
\end{tabular}




\newpage
%%%%%%%%%%%%%%%%%%%%%%%%%%%%%
\[
  p = \frac{h\nu}{c} = \frac{h}{\lambda}
\]


\vskip3mm
\noindent
【光子の運動量】{\footnotesize (p. 186)}

\begin{tabular}{ccl}
$p$	&[kg$\cdot$m/s]	&:光子の運動量 \\
$h$	&{[J$\cdot$s]}	&:プランク定数 \\
$\nu$	&[Hz]	&:光の振動数 \\
$c$	&[m/s]	&:光速 \\
$\lambda$	&[m]	&:光の波長 \\
\end{tabular}





\newpage
%%%%%%%%%%%%%%%%%%%%%%%%%%%%%
\[
  \frac{1}{\lambda} = R \left(\frac{1}{m^2} - \frac{1}{n^2} \right)
\]


\vskip3mm
\noindent
【リュードベリの公式】{\footnotesize (p. 189)}

\begin{tabular}{ccl}
$\lambda$	&[m]	&:波長 \\
  $R$	&[m$^{-1}$]	&:リュードベリ定数 \\
$m, n$	&[--]	&:整数
\end{tabular}





\newpage
%%%%%%%%%%%%%%%%%%%%%%%%%%%%%
\[
  2 \pi r \times p = nh
\]


\vskip3mm
\noindent
【量子条件】{\footnotesize (p. 191)}

\begin{tabular}{ccl}
$r$	&[m]	&:軌道半径 \\
$p$	&[kg$\cdot$m/s]	&:運動量 \\
$n$	&[--]	&:整数 \\
$h$	&{[J$\cdot$s]}	&:プランク定数
\end{tabular}



\newpage
%%%%%%%%%%%%%%%%%%%%%%%%%%%%%
\[
  2d \sin\theta = m\lambda
  \]


\vskip3mm
\noindent
【ブラッグの法則】{\footnotesize (p. 197)}

\begin{tabular}{ccl}
$d$	&[m]	&:原子の間隔 \\
$\theta$	&[$^\circ$]	&:角度 \\
$m$	&[--]	&:整数 \\
$\lambda$	&{[m]}	&:X線の波長
\end{tabular}




\newpage
%%%%%%%%%%%%%%%%%%%%%%%%%%%%%
\[
  \lambda = \frac{h}{p} = \frac{h}{mv}
\]


\vskip3mm
\noindent
【ドブロイ波長】{\footnotesize (p. 199)}

\begin{tabular}{ccl}
$\lambda$	&[m]	&:ドブロイ波長 \\
$h$	&{[J$\cdot$s]}	&:プランク定数 \\
$p$	&[kg$\cdot$m/s]	&:運動量 \\
$m$	&[kg]	&:質量 \\
$v$	&[m/s]	&:速度
\end{tabular}






\newpage
%%%%%%%%%%%%%%%%%%%%%%%%%%%%%
\[
  N(t) = N_0 \left( \frac{1}{2} \right)^{t/T}
\]

\vskip3mm
\noindent
【半減期】{\footnotesize (p. 209)}

\begin{tabular}{ccl}
$N$	&[個]	&:原子核の個数 \\
$N_0$	&[個]	&:初めの原子核の個数 \\
$t$	&[s]	&:時刻 \\
$T$	&[s]	&:半減期
\end{tabular}





\newpage
%%%%%%%%%%%%%%%%%%%%%%%%%%%%%
\[
  \mathit{\Delta} E = \mathit{\Delta}M c^2
\]
\vskip3mm
\noindent
\resizebox{\textwidth}{1zh}{【アインシュタインの関係式】{\footnotesize (p. 210)}}

\begin{tabular}{ccl}
  $\mathit{\Delta}E$	&[J]	&:\resizebox{27mm}{1zh}{エネルギーの変化量} \\
$\mathit{\Delta}M$	&[kg]	&:質量の変化量 \\
$c$	&[m/s]	&:光速
\end{tabular}




\end{document}
