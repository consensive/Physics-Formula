\documentclass[10pt]{jarticle}

\renewcommand{\contentsname}{波動}
\usepackage{kousiki}


\begin{document}
%%%%%%%%%%%%%%%%%%%%%%%%%%%%%
%Foreword Page
\addtocounter{page}{-1}
\thispagestyle{empty}

まえがき\\

{\scriptsize
『初歩から学ぶ基礎物理学 熱・波動』(第日本図書)の波動分野で現れる法則・公式をまとめました.//

 演習は単なる算数ではなく思考の実体験の場です.
意味記憶だけではなく,エピソード記憶として法則・公式を自身の思考に取り入れてもらえることを願っています.\\

 習得してから公式集を振り返ると,物理教師がよく言う「公式は暗記するものではない.理解するものだ」という台詞の意味を実感してもらえるはずです.\\




\hfill
釧路高専(物理)

\hfill
松崎俊明

\vfill
\hfill
(2016-12-15)
}


%%%%%%%%%%%%%%%%%%%%%%%%%%%%%
%TOC Page
\addtocounter{page}{-1}
\thispagestyle{empty}
\tableofcontents



\newpage
%%%%%%%%%%%%%%%%%%%%%%%%%%%%%
%Title Page
\addtocounter{page}{-1}
\thispagestyle{empty}
\section{波}

\begin{enumerate}
\small
\itemsep-4mm
\item 振動数と周期\\
\item 波の基本式\\
\item 重ね合わせの原理\\
\item 強めあう点の条件\\
\item 屈折率の式\\
\end{enumerate}
%%%%%%%%%%%%%%%%%%%%%%%%%%%%%




\newpage
%%%%%%%%%%%%%%%%%%%%%%%%%%%%%
\[
T = \frac{1}{\; f \;}
\]


\vskip3mm
【振動数と周期】{\footnotesize (p. 107)}

\begin{tabular}{ccl}
$T$	&[s]	&:周期 \\
$f$	& [Hz]	&:振動数
\end{tabular}

\newpage



%%%%%%%%%%%%%%%%%%%%%%%%%%%%%
\[
v = \frac{\lambda}{\; T \;} = f \lambda
\]


\vskip3mm
【波の基本式】{\footnotesize (p. 108)}

\begin{tabular}{ccl}
$v$	& [m/s]	&:波の速さ\\
$\lambda$	& [m]	&:波長\\
$f$	& [Hz]	&:振動数\\
$T$	&[s]	&:周期
\end{tabular}

\newpage




%%%%%%%%%%%%%%%%%%%%%%%%%%%%%
\[
y = y_\mathrm{A} + y_\mathrm{B}
\]


\vskip3mm
【重ね合わせの原理】{\footnotesize (p. 114)}

\begin{tabular}{ccl}
$y$	& [m]	&:合成波の変位\\
$y_\mathrm{A}$	& [m]	&:波源Aからの変位\\
$y_\mathrm{B}$	& [m]	&:波源Bからの変位
\end{tabular}

\newpage


%%%%%%%%%%%%%%%%%%%%%%%%%%%%%
\[
| l_1 - l_2 |= m \lambda
\]


\vskip3mm
【強めあう点の条件】{\footnotesize (p. 120)}

\begin{tabular}{ccl}
$l_i$	& [m]	&:波源からの距離\\
$m$	& 	&:整数\\
$\lambda$	& [m]	&:波長
\end{tabular}

\newpage




%%%%%%%%%%%%%%%%%%%%%%%%%%%%%
\[
n_{12}	= \frac{v_1}{\; v_2 \;}
	= \frac{\lambda_1}{\; \lambda_2 \;}
	= \frac{\sin i}{\; \sin r \;} \\
\]


\vskip3mm
【屈折率の式】{\footnotesize (p. 125)}

\begin{tabular}{ccl}
$n_{12}$	&	&:相対屈折率\\
$v$	& [m/s]	&:波の速さ\\
$\lambda$	&[m]	&:波長\\
$i$	&[°]	&:入射角\\
$r$	&[°]	&:屈折角\\
\end{tabular}




\newpage
%%%%%%%%%%%%%%%%%%%%%%%%%%%%%
%Title Page
\addtocounter{page}{-1}
\thispagestyle{empty}
\section{音波}

\begin{enumerate}
\setcounter{enumi}{\thepage}
\small
\itemsep-4mm
\item 空気中の音速\\
\item うなりの式\\
\item 弦の固有振動数\\
\item 弦の固有振動数\\
\item 開管の固有振動数\\
\item 閉管の固有振動数\\
\item ドップラー効果\\
\end{enumerate}
%%%%%%%%%%%%%%%%%%%%%%%%%%%%%
\newpage



%%%%%%%%%%%%%%%%%%%%%%%%%%%%%
\[
V = 331.5 + 0.6t
\]


\vskip3mm
【空気中の音速】{\footnotesize (p. 131)}

\begin{tabular}{ccl}
$V$	&[m/s]	&:空気中の音速\\
$t$	&[$^\circ$C]	&:空気の温度\\
\end{tabular}

\newpage





%%%%%%%%%%%%%%%%%%%%%%%%%%%%%
\[
f = |f_2 - f_1|
\]


\vskip3mm
【うなりの式】{\footnotesize (p. 136)}

\begin{tabular}{ccl}
$f$	&[Hz]	&:うなりの振動数\\
$f_i$	&[Hz]	&:振動数\\
\end{tabular}

\newpage




%%%%%%%%%%%%%%%%%%%%%%%%%%%%%
\[
f_m = \frac{V}{\; 2l \;}\times m
\]


\vskip3mm
【弦の固有振動数】{\footnotesize (p. 138)}

\begin{tabular}{ccl}
$f_m$	&[Hz]	&:固有振動数\\
$V$	&[m/s]	&:弦を伝わる音速\\
$l$	&[m]	&:弦の長さ\\
$m$	&	&:自然数\\
\end{tabular}

\newpage




%%%%%%%%%%%%%%%%%%%%%%%%%%%%%
\[
V = \sqrt{\,\, \frac{S}{\;\; \rho \;\;}\,\,}
\]


\vskip3mm
【弦の固有振動数】{\footnotesize (p. 139)}

\begin{tabular}{ccl}
$V$	&[m/s]	&:弦を伝わる音速\\
$S$	&[N]	&:張力\\
$\rho$	&[kg/m]	&:線密度\\
\end{tabular}

\newpage





%%%%%%%%%%%%%%%%%%%%%%%%%%%%%
\[
f_m = \frac{V}{\; 2L \;}\times m
\]


\vskip3mm
【開管の固有振動数】{\footnotesize (p. 141)}

\begin{tabular}{ccl}
$f_m$	&[Hz]	&:固有振動数\\
$V$	&[m/s]	&:音速\\
$L$	&[m]	&:気柱の長さ\\
$m$	&	&:自然数\\
\end{tabular}

\newpage





%%%%%%%%%%%%%%%%%%%%%%%%%%%%%
\[
f_m = \frac{V}{\; 4L \;}\times m
\]


\vskip3mm
【閉管の固有振動数】{\footnotesize (p. 142)}

\begin{tabular}{ccl}
$f_m$	&[Hz]	&:固有振動数\\
$V$	&[m/s]	&:音速\\
$L$	&[m]	&:気柱の長さ\\
$m$	&	&:奇数\\
\end{tabular}

\newpage






%%%%%%%%%%%%%%%%%%%%%%%%%%%%%
\[
f = \frac{V-u}{V-v}\, f_0
\]


\vskip3mm
【ドップラー効果】{\footnotesize (p. 146)}

\begin{tabular}{ccl}
$f$	&[Hz]	&:観測される振動数\\
$V$	&[m/s]	&:音速\\
$u$	&[m/s]	&:観測者の速度\\
$v$	&[m/s]	&:音源の速度\\
$f_0$	&[Hz]	&:音源の振動数\\
\end{tabular}

\newpage




\newpage
%%%%%%%%%%%%%%%%%%%%%%%%%%%%%
%Title Page
\addtocounter{page}{-1}
\thispagestyle{empty}
\section{光波}

\begin{enumerate}
\setcounter{enumi}{\thepage}
\small
\itemsep-4mm
\item 絶対屈折率\\
\item 絶対屈折率と相対屈折率\\
\item ヤングの実験\\
\item 回折格子の実験\\
\item ニュートンリング\\
\item 鏡の公式\\
\item レンズの公式\\
\end{enumerate}
%%%%%%%%%%%%%%%%%%%%%%%%%%%%%
\newpage




%%%%%%%%%%%%%%%%%%%%%%%%%%%%%
\[
n_1 = \frac{c}{\; v_1 \;}
\]


\vskip3mm
【絶対屈折率】{\footnotesize (p. 150)}

\begin{tabular}{ccl}
$n_1$	&	&:絶対屈折率\\
$c$	& [m/s]	&:真空中での光速\\
$v_1$	& [m/s]	&:物質1中での光速\\
\end{tabular}

\newpage

%%%%%%%%%%%%%%%%%%%%%%%%%%%%%
\[
n_{12} = \frac{n_2}{\; n_1 \;}
\]


\vskip3mm
【相対屈折率と絶対屈折率】\\
\hfill {\footnotesize (p. 150)}

\begin{tabular}{ccl}
$n_{12}$	&	&:相対屈折率\\
$n_i$	&	&:絶対屈折率\\
\end{tabular}

\newpage






%%%%%%%%%%%%%%%%%%%%%%%%%%%%%
\[
\frac{\; xd \;}{L} = m \lambda
\]


\vskip3mm
【ヤングの実験】{\footnotesize (p. 154)}

\begin{tabular}{ccl}
$x$	&[m]	&:明点の位置\\
$d$	&[m]	&:スリットの間隔\\
$L$	&[m]	&:{\small スクリーンまでの距離}\\
$m$	&	&:整数\\
$\lambda$	&[m]	&:波長
\end{tabular}

\newpage




%%%%%%%%%%%%%%%%%%%%%%%%%%%%%
\[
d \sin\theta = m \lambda
\]


\vskip3mm
【回折格子の実験】{\footnotesize (p. 155)}

\begin{tabular}{ccl}
$d$	&[m]	&:格子定数\\
$\theta$	&[rad]	&:角度\\
$m$	&	&:整数\\
$\lambda$	&[m]	&:波長\\
\end{tabular}

\newpage




%%%%%%%%%%%%%%%%%%%%%%%%%%%%%
\[
\frac{\, r^2\, }{R \;} = m \lambda
\]


\vskip3mm
【ニュートンリング】{\footnotesize (p. 157)}

\begin{tabular}{ccl}
$r$	&[m]	&:暗線リングの半径\\
$R$	&[m]	&:レンズの球面半径\\
$m$	&	&:整数\\
$\lambda$	&[m]	&:波長\\
\end{tabular}

\newpage





%%%%%%%%%%%%%%%%%%%%%%%%%%%%%
\[
\frac{1}{\; a \;} + \frac{1}{\; b \;} = \frac{1}{\; f \;}
\]
\[
f = \frac{\; R \;}{2}
\]


\vskip3mm
【鏡の公式】{\footnotesize (p. 167)}

\begin{tabular}{ccl}
$a$	&[m]	&:物体の位置\\
$b$	&[m]	&:像の位置\\
$f$	&[m]	&:焦点距離\\
$R$	&[m]	&:鏡面の半径\
\end{tabular}

\newpage





%%%%%%%%%%%%%%%%%%%%%%%%%%%%%
\[
\frac{1}{\; a \;} + \frac{1}{\; b \;} = \frac{1}{\; f \;}
\]


\vskip3mm
【レンズの公式】{\footnotesize (p. 171)}

\begin{tabular}{ccl}
$a$	&[m]	&:物体の位置\\
$b$	&[m]	&:像の位置\\
$f$	&[m]	&:焦点距離
\end{tabular}

\newpage





\end{document}
