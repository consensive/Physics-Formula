\documentclass[10pt]{jarticle}

\renewcommand{\contentsname}{熱}
\usepackage{kousiki}


\begin{document}


%%%%%%%%%%%%%%%%%%%%%%%%%%%%%
%TOC Page
\addtocounter{page}{-1}
\thispagestyle{empty}
\tableofcontents



\newpage
%%%%%%%%%%%%%%%%%%%%%%%%%%%%%
%Title Page
\addtocounter{page}{-1}
\thispagestyle{empty}
\section{熱エネルギー}


\begin{enumerate}
\small
\itemsep-4mm
\item 摂氏と絶対温度\\
\item 熱の仕事当量\\
\item 熱容量と温度変化\\
\item 熱容量と比熱\\
\item 線膨張率\\
\item 体積膨張率\\
\item 熱伝導率
\end{enumerate}


\newpage
%%%%%%%%%%%%%%%%%%%%%%%%%%%%%
\[
	T = t + 273.15
\]


\vskip3mm
【摂氏と絶対温度】{\footnotesize (p. 11)}

\begin{tabular}{ccl}
$T$	&[K]	&:絶対温度\\
$t$	&[$^\circ$C]	&:摂氏温度
\end{tabular}



\newpage
%%%%%%%%%%%%%%%%%%%%%%%%%%%%%
\[
	J =4.19 \mathrm{(J / cal)}
\]


\vskip3mm
【熱の仕事当量】{\footnotesize (p. 12)}




\newpage
%%%%%%%%%%%%%%%%%%%%%%%%%%%%%
\[
	Q = C \mathit{\Delta} T
\]


\vskip3mm
【熱容量と温度変化】{\footnotesize (p. 14)}

\begin{tabular}{ccl}
$Q$	&[J]	&:熱量\\
$C$	&[J/K]	&:熱容量\\
$\mathit{\Delta} T$	& [K]	&:温度変化
\end{tabular}

\newpage





\newpage
%%%%%%%%%%%%%%%%%%%%%%%%%%%%%
\[
	C = m c
\]


\vskip3mm
【熱容量と比熱】{\footnotesize (p. 15)}

\begin{tabular}{ccl}
$C$	&[J/K]	&:熱容量\\
$m$	&[g]	&:質量\\
$c$	&[J/g $\!\! \cdot \!\! $ K]	&:比熱
\end{tabular}

\newpage




\newpage
%%%%%%%%%%%%%%%%%%%%%%%%%%%%%
\[
	l = l_0 (1+ \alpha \mathit{\Delta}T)
\]


\vskip3mm
【線膨張率】{\footnotesize (p. 23)}

\begin{tabular}{ccl}
$l$	&[m]	&:変化後の長さ\\
$l_0$	&[m]	&:元の長さ\\
$\alpha$	&[K$^{-1}$]	&:線膨張率\\
$\mathit{\Delta} T$	& [K]	&:温度変化
\end{tabular}



\newpage
%%%%%%%%%%%%%%%%%%%%%%%%%%%%%
\[
	V = V_0 (1+ \beta \mathit{\Delta}T)
\]


\vskip3mm
【体積膨張率】{\footnotesize (p. 23)}

\begin{tabular}{ccl}
$V$	&[m$^3$]	&:変化後の長さ\\
$V_0$	&[m$^3$]	&:元の長さ\\
$\beta$	&[K$^{-1}$]	&:体積膨張率\\
$\mathit{\Delta} T$	& [K]	&:温度変化
\end{tabular}




\newpage
%%%%%%%%%%%%%%%%%%%%%%%%%%%%%
\[
	Q = \kappa S \frac{\mathit{\Delta}T}{l}
\]


\vskip3mm
【熱伝導率】{\footnotesize (p. 26)}

\begin{tabular}{ccl}
$Q$	&[W]	&:流れる熱量\\
$\kappa$	&[W/m$\cdot$K]	&:熱伝導率\\
$\mathit{\Delta} T$	& [K]	&:温度差\\
$l$	&[m]	&:長さ
\end{tabular}



\newpage
%%%%%%%%%%%%%%%%%%%%%%%%%%%%%
%Title Page
\addtocounter{page}{-1}
\thispagestyle{empty}
\section{気体}


\begin{enumerate}
\setcounter{enumi}{\thepage}
\small
\itemsep-4mm
\item 圧力\\
\item ボイル・シャルルの法則\\
\item 理想気体の状態方程式\\
\item 分子による圧力\\
\item 単原子分子の内部エネルギー\\
\end{enumerate}






\newpage
%%%%%%%%%%%%%%%%%%%%%%%%%%%%%
\[
	p = \frac{F}{\; S \;}
\]


\vskip3mm
【圧力】{\footnotesize (p. 30)}

\begin{tabular}{ccl}
$p$	&[Pa]	&:圧力\\
$F$	&[N]	&:力\\
$S$	&[m$^2$]	&:面積
\end{tabular}



\newpage
%%%%%%%%%%%%%%%%%%%%%%%%%%%%%
\[
	\frac{\; pV \;}{T}= (一定)
\]


\vskip3mm
【ボイル・シャルルの法則】\\
\hfill {\footnotesize (p. 35)}

\begin{tabular}{ccl}
$p$	&[Pa]	&:圧力\\
$V$	&[m$^3$]	&:体積\\
$T$	&[K]	&:絶対温度
\end{tabular}




\newpage
%%%%%%%%%%%%%%%%%%%%%%%%%%%%%
\[
	p V = n R T
\]


\vskip3mm
【理想気体の状態方程式】\\
\hfill {\footnotesize (p. 38)}

\begin{tabular}{ccl}
$p$	&[Pa]	&:圧力\\
$V$	&[m$^3$]	&:体積\\
$n$	&[mol]	&:モル数\\
$R$	&[J/mol $\!\! \cdot \!\! $ K]	&:気体定数\\
$T$	&[K]	&:絶対温度
\end{tabular}





\newpage
%%%%%%%%%%%%%%%%%%%%%%%%%%%%%
\[
	p = \frac{N m \overline{v^2}}{3V}
\]


\vskip3mm
【分子による圧力】{\footnotesize (p. 43)}

\begin{tabular}{ccl}
$p$	&[Pa]	&:圧力\\
$N$	&[個]	&:分子の個数\\
$m$	&[kg]	&:分子一つの質量\\
$\overline{v^2}$	&[(m/s)$^2$]	&:{\small 速度の二乗の平均}\\
$V$	&[m$^3$]	&:体積
\end{tabular}






\newpage
%%%%%%%%%%%%%%%%%%%%%%%%%%%%%
\[
	U = \frac{3}{\; 2 \;} n RT
\]


\vskip3mm
【単原子分子の内部エネルギー】\\
\hfill {\footnotesize (p. 48)}

\begin{tabular}{ccl}
$U$	&[J]	&:内部エネルギー\\
$n$	&[mol]	&:モル数\\
$R$	&[J/mol $\!\! \cdot \!\! $ K]	&:気体定数\\
$T$	&[K]	&:絶対温度
\end{tabular}




\newpage
%%%%%%%%%%%%%%%%%%%%%%%%%%%%%
%Title Page
\addtocounter{page}{-1}
\thispagestyle{empty}
\section{熱力学}


\begin{enumerate}
\setcounter{enumi}{\thepage}
\small
\itemsep-4mm
\item 熱力学第1法則\\
\item 気体が外部にした仕事\\
\item ポアソンの法則\\
\item マイヤーの関係式\\
\item 熱機関の効率\\
\item エントロピー\\
\item カルノーサイクルの効率
\end{enumerate}






\newpage
%%%%%%%%%%%%%%%%%%%%%%%%%%%%%
\[
	\mathit{\Delta} U = Q + W
\]


\vskip3mm
【熱力学第1法則】{\footnotesize (p. 53)}

\begin{tabular}{ccl}
$\mathit{\Delta} U$	&[J]	&:{\small 内部エネルギーの変化}\\
$Q$	&[J]	&:外部から吸収した熱\\
$W$	&[J]	&:外部からされた仕事
\end{tabular}





\newpage
%%%%%%%%%%%%%%%%%%%%%%%%%%%%%
\[
	W =  p \mathit{\Delta}V = \int \! \! p \, dV
\]


\vskip3mm
【気体が外部にした仕事】\\
\hfill {\footnotesize (p. 55, 57)}

\begin{tabular}{ccl}
$W$	&[J]	&:{\small 気体がした仕事}\\
$p$	&[Pa]	&:気体の圧力\\
$\mathit{\Delta} V$	&[m$^3$]	&:気体の体積変化
\end{tabular}





\newpage
%%%%%%%%%%%%%%%%%%%%%%%%%%%%%
\[
	pV^\gamma = (一定)
\]


\vskip3mm
【ポアソンの法則】{\footnotesize (p. 63)}

\begin{tabular}{ccl}
$p$	&[Pa]	&:圧力\\
$V$	&[m$^3$]	&:体積\\
$\gamma$	&	&:比熱比 ($= C_p / C_V$)
\end{tabular}




\newpage
%%%%%%%%%%%%%%%%%%%%%%%%%%%%%
\[
	C_p = C_V + R
\]


\vskip3mm
【マイヤーの関係式】{\footnotesize (p. 66)}

\begin{tabular}{ccl}
$C_p$	&[J/mol $\!\! \cdot \!\! $ K]	&:定圧比熱\\
$C_V$	&[J/mol $\!\! \cdot \!\! $ K]	&:定積比熱\\
$R$	&[J/mol $\!\! \cdot \!\! $ K]	&:気体定数\
\end{tabular}




\newpage
%%%%%%%%%%%%%%%%%%%%%%%%%%%%%
\[
	\eta = \frac{W}{Q_\mathrm{H}} = 1- \frac{Q_\mathrm{L}}{Q_\mathrm{H}}
\]


\vskip3mm
【熱機関の効率】{\footnotesize (p. 72)}

\begin{tabular}{ccl}
$\eta$	&	&:熱効率\\
$W$	&[J]	&:熱機関がした仕事\\
$Q_\mathrm{H}$	&[J]	&:受け取った熱\\
$Q_\mathrm{L}$	&[J]	&:放出した熱
\end{tabular}



\newpage
%%%%%%%%%%%%%%%%%%%%%%%%%%%%%
\[
	\mathit{\Delta} S = \frac{\mathit{\Delta} Q}{T}
\]


\vskip3mm
【エントロピー】{\footnotesize (p. 77)}

\begin{tabular}{ccl}
$\mathit{\Delta} S$	&[J/K]	&:エントロピー変化\\
$\mathit{\Delta} Q$	&[J]	&:系が得た熱\\
$T$	&[K]	&:温度
\end{tabular}




\newpage
%%%%%%%%%%%%%%%%%%%%%%%%%%%%%
\[
	\eta_\mathrm{c} =  1- \frac{\; T_\mathrm{L}}{\; T_\mathrm{H}}
\]


\vskip3mm
【カルノーサイクルの効率】\\
\hfill {\footnotesize (p. 80)}

\begin{tabular}{ccl}
$\eta_\mathrm{c}$	&	&:熱効率\\
$T_\mathrm{L}$	&[J]	&:低温熱源の温度\\
$T_\mathrm{H}$	&[J]	&:高温熱源の温度
\end{tabular}


\end{document}
