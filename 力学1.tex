\documentclass[10pt]{jarticle}


\renewcommand{\contentsname}{力学 I}
\usepackage{kousiki}


\begin{document}


%%%%%%%%%%%%%%%%%%%%%%%%%%%%%
%TOC Page
\addtocounter{page}{-1}
\thispagestyle{empty}
\tableofcontents



\newpage
%%%%%%%%%%%%%%%%%%%%%%%%%%%%%
%Title Page
\addtocounter{page}{-1}
\thispagestyle{empty}
\section{物体の運動}


\begin{enumerate}
\small
\itemsep-4mm
\item 速度の定義\\
\item 加速度の定義\\
\item 等加速度運動の速度\\
\item 等加速度運動の位置\\
\item 便利な式\\
\item 合成速度\\
\item 重力加速度の値
\end{enumerate}
%%%%%%%%%%%%%%%%%%%%%%%%%%%%%


\newpage
%%%%%%%%%%%%%%%%%%%%%%%%%%%%%
\[
	v = \frac{\mathit{\Delta} x}{\mathit{\Delta} t}
\]


\vskip3mm
【速度の定義】{\footnotesize (p. 10)}

\begin{tabular}{ccl}
$v$	&[m/s]	&:速度\\
$\mathit{\Delta} x$	&[m]	&:移動距離\\
$\mathit{\Delta} t$	&[s]	&:経過時間
\end{tabular}



\newpage
%%%%%%%%%%%%%%%%%%%%%%%%%%%%%
%p. 13
\[
	a = \frac{\mathit{\Delta} v}{\mathit{\Delta} t}
\]


\vskip3mm
【加速度の定義】{\footnotesize (p. 13)}

\begin{tabular}{ccl}
$a$	&[m/s$^2$]	&:加速度\\
$\mathit{\Delta} v$	&[m/s]	&:速度の変化\\
$\mathit{\Delta} t$	&[s]	&:経過時間
\end{tabular}




\newpage
%%%%%%%%%%%%%%%%%%%%%%%%%%%%%
%p. 16
\[
	v = v_0 + a t
\]


\vskip3mm
【等加速度運動の速度】{\footnotesize (p. 16)}

\begin{tabular}{ccl}
$v$	&[m/s]	&:速度\\
$v_0$	&[m/s]	&:初速度\\
$a$	&[m/s$^2$]	&:加速度\\
$t$	&[s]	&:時間
\end{tabular}




\newpage
%%%%%%%%%%%%%%%%%%%%%%%%%%%%%
%p. 16
\[
	x = v_0\, t + \frac{1}{2} a t^2
\]


\vskip3mm
【等加速度運動の位置】{\footnotesize (p. 16)}

\begin{tabular}{ccl}
$x$	&[m]	&:位置\\
$v_0$	&[m/s]	&:初速度\\
$t$	&[s]	&:時間\\
$a$	&[m/s$^2$]	&:加速度
\end{tabular}





\newpage
%%%%%%%%%%%%%%%%%%%%%%%%%%%%%
%p. 17
\[
	v^2 - v_0^2 = 2 a x
\]


\vskip3mm
【便利な式】{\footnotesize (p. 17)}

\begin{tabular}{ccl}
$v$	&[m/s]	&:速度\\
$v_0$	&[m/s]	&:初速度\\
$a$	&[m/s$^2$]	&:加速度\\
$x$	&[m]	&:位置
\end{tabular}






\newpage
%%%%%%%%%%%%%%%%%%%%%%%%%%%%%
%p. 20
\[
	\vec{V} = \vec{u} + \vec{v}
\]


\vskip3mm
【合成速度】{\footnotesize (p. 20)}

\begin{tabular}{ccl}
$\vec{V}$	&[m/s]	&\small :Oから見たBの速度\\
$\vec{u}$	&[m/s]	&\small :AからみたBの速度\\
$\vec{v}$	&[m/s]	&\small :OからみたAの速度
\end{tabular}





\newpage
%%%%%%%%%%%%%%%%%%%%%%%%%%%%%
%p. 25
\[
	g = 9.8  \mathrm{(m/s^2)}
\]


\vskip3mm
【重力加速度の値】{\footnotesize (p. 25)}







\newpage
%%%%%%%%%%%%%%%%%%%%%%%%%%%%%
%Title Page
\addtocounter{page}{-1}
\thispagestyle{empty}
\section{力と運動}

\begin{enumerate}
\setcounter{enumi}{\thepage}
\small
\itemsep-4mm
\item 力の合成\\
\item 重力\\
\item フックの法則\\
\item 最大摩擦力\\
\item 動摩擦力\\
\item 運動方程式
\end{enumerate}
%%%%%%%%%%%%%%%%%%%%%%%%%%%%%



\newpage
%%%%%%%%%%%%%%%%%%%%%%%%%%%%%
%p. 35
\[
	\vec{F}_{合} = \vec{F}_\mathrm{A} + \vec{F}_\mathrm{B}
\]


\vskip3mm
【力の合成】{\footnotesize (p. 35)}

\begin{tabular}{ccl}
$\vec{F}_{合}$	&[N]	&:合力\\
$\vec{F}_{A}$	&[N]	&:一つ目の力\\
$\vec{F}_{B}$	&[N]	&:二つ目の力
\end{tabular}





\newpage
%%%%%%%%%%%%%%%%%%%%%%%%%%%%%
%p. 44
\[
	W = mg
\]


\vskip3mm
【重力】{\footnotesize (p. 44)}

\begin{tabular}{ccl}
$W$	&[N]	&:重力\\
$m$	&[kg]	&:質量\\
$g$	&[m/s$^2$]	&:重力加速度
\end{tabular}






\newpage
%%%%%%%%%%%%%%%%%%%%%%%%%%%%%
%p. 47
\[
	F = k x
\]


\vskip3mm
【フックの法則】{\footnotesize (p. 47)}

\begin{tabular}{ccl}
$F$	&[N]	&:弾性力\\
$k$	&[N/m]	&:ばね定数\\
$x$	&[m]	&:ばねの伸び
\end{tabular}






\newpage
%%%%%%%%%%%%%%%%%%%%%%%%%%%%%
%p. 50
\[
	F_\mathrm{max} =  \mu N
\]


\vskip3mm
【最大摩擦力】{\footnotesize (p. 50)}

\begin{tabular}{ccl}
$F_\mathrm{max}$	&[N]	&:最大摩擦力\\
$\mu$	&	&:静止摩擦係数\\
$N$	&[N]	&:垂直効力
\end{tabular}





\newpage
%%%%%%%%%%%%%%%%%%%%%%%%%%%%%
%p. 52
\[
	f' =  \mu' N
\]


\vskip3mm
【動摩擦力】{\footnotesize (p. 50)}

\begin{tabular}{ccl}
$f'$	&[N]	&:動摩擦力\\
$\mu'$	&	&:動摩擦係数\\
$N$	&[N]	&:垂直効力
\end{tabular}





\newpage
%%%%%%%%%%%%%%%%%%%%%%%%%%%%%
%p. 60
\[
	m a = F
\]


\vskip3mm
【運動方程式】{\footnotesize (p. 60)}

\begin{tabular}{ccl}
$m$	&[kg]	&:質量\\
$a$	&[m/s$^2$]	&:加速度\\
$F$	&[N]	&:力
\end{tabular}





\newpage
%%%%%%%%%%%%%%%%%%%%%%%%%%%%%
%Title Page
\addtocounter{page}{-1}
\thispagestyle{empty}
\section{運動量保存則}

\begin{enumerate}
\setcounter{enumi}{\thepage}
\small
\itemsep-4mm
\item 運動量と力積の変化\\
\item 力積の定義\\
\item 運動量の定義\\
\item 運動量保存則\\
\item 反発係数
\end{enumerate}
%%%%%%%%%%%%%%%%%%%%%%%%%%%%%




\newpage
%%%%%%%%%%%%%%%%%%%%%%%%%%%%%
%p. 76
\[
	mv - mv_0 = F \mathit{\Delta}t
\]


\vskip3mm
【運動量と力積の変化】{\footnotesize (p. 76)}

\begin{tabular}{ccl}
$m$	&[kg]	&:質量\\
$v$	&[m/s]	&:速度\\
$v_0$	&[m/s]	&:初速度\\
$F$	&[N]	&:力\\
$\mathit{\Delta}t$	&[s]	&:時間
\end{tabular}






\newpage
%%%%%%%%%%%%%%%%%%%%%%%%%%%%%
%p. 77
\[
	I = F \mathit{\Delta}t
\]


\vskip3mm
【力積の定義】{\footnotesize (p. 77)}

\begin{tabular}{ccl}
$I$	&[N $\!\! \cdot \!\!$ s]	&:力積\\
$F$	&[N]	&:力\\
$\mathit{\Delta}t$	&[s]	&:時間
\end{tabular}





\newpage
%%%%%%%%%%%%%%%%%%%%%%%%%%%%%
%p. 78
\[
	p = mv
\]


\vskip3mm
【運動量の定義】{\footnotesize (p. 78)}

\begin{tabular}{ccl}
$p$	&[kg $\!\! \cdot \!\!$ m/s]	&:運動量\\
$m$	&[kg]	&:質量\\
$v$	&[m/s]	&:速度
\end{tabular}






\newpage
%%%%%%%%%%%%%%%%%%%%%%%%%%%%%
%p. 81
\[
	\Sigma \, \vec{p_i} = \Sigma \, \vec{p'}\!_i
\]

\vskip3mm
【運動量保存則】{\footnotesize (p. 81)}

\begin{tabular}{ccl}
$\vec{p}_i$	&[kg $\!\! \cdot \!\!$ m/s]	&:衝突前の運動量\\
$\vec{p}_i$	&[kg $\!\! \cdot \!\!$ m/s]	&:衝突後の運動量
\end{tabular}





\newpage
%%%%%%%%%%%%%%%%%%%%%%%%%%%%%
\[
	e = \frac{|\vec{v}'|}{|\vec{v}|}
\]


\vskip3mm
【反発係数】{\footnotesize (p. 87)}

\begin{tabular}{ccl}
$e$	&	&:反発係数\\
$|\vec{v}'|$	&[m/s]	&:衝突後の速さ\\
$|\vec{v}|$	&[m/s]	&:衝突前の速さ
\end{tabular}





\newpage
%%%%%%%%%%%%%%%%%%%%%%%%%%%%%
%Title Page
\addtocounter{page}{-1}
\thispagestyle{empty}
\section{力学的エネルギー}

\begin{enumerate}
\setcounter{enumi}{\thepage}
\small
\itemsep-4mm
\item 仕事の定義\\
\item 仕事率の定義\\
\item 運動エネルギー\\
\item 重力による位置エネルギー\\
\item 弾性エネルギー\\
\item エネルギー保存則
\end{enumerate}
%%%%%%%%%%%%%%%%%%%%%%%%%%%%%




\newpage
%%%%%%%%%%%%%%%%%%%%%%%%%%%%%
\[
	W = F x
\]


\vskip3mm
【仕事の定義】{\footnotesize (p. 94)}

\begin{tabular}{ccl}
$W$	&[J]	&:仕事\\
$F$	&[N]	&:力\\
$x$	&[m]	&:移動距離
\end{tabular}





\newpage
%%%%%%%%%%%%%%%%%%%%%%%%%%%%%
\[
	P = \frac{W}{t}
\]


\vskip3mm
【仕事率の定義】{\footnotesize (p. 98)}

\begin{tabular}{ccl}
$P$	&[W]	&:仕事率\\
$W$	&[J]	&:仕事\\
$t$	&[s]	&:時間
\end{tabular}




\newpage
%%%%%%%%%%%%%%%%%%%%%%%%%%%%%
\[
	K = \frac{1}{2} m v^2
\]


\vskip3mm
【運動エネルギー】{\footnotesize (p. 102)}

\begin{tabular}{ccl}
$K$	&[J]	&:運動エネルギー\\
$m$	&[kg]	&:質量\\
$v$	&[m/s]	&:速度
\end{tabular}





\newpage
%%%%%%%%%%%%%%%%%%%%%%%%%%%%%
\[
	U = mgh
\]


\vskip3mm
【重力による位置エネルギー】\\
\hfill{\footnotesize (p. 104)}

\begin{tabular}{ccl}
$U$	&[J]	&:位置エネルギー\\
$m$	&[kg]	&:質量\\
$g$	&[m/s$^2$]	&:重力加速度\\
$h$	&[m]	&:高さ
\end{tabular}





\newpage
%%%%%%%%%%%%%%%%%%%%%%%%%%%%%
\[
	U = \frac{1}{2} kx^2
\]


\vskip3mm
【弾性エネルギー】{\footnotesize (p. 106)}

\begin{tabular}{ccl}
$U$	&[J]	&:弾性エネルギー\\
$k$	&[N/m]	&:ばね定数\\
$x$	&[m]	&:ばねの伸び
\end{tabular}





\newpage
%%%%%%%%%%%%%%%%%%%%%%%%%%%%%
\[
	K + U = K' + U'
\]


\vskip3mm
【エネルギー保存則】{\footnotesize (p. 108)}

\begin{tabular}{ccl}
$K$	&[J]	&:運動エネルギー\\
$U$	&[J]	&:位置エネルギー
\end{tabular}





\newpage
%%%%%%%%%%%%%%%%%%%%%%%%%%%%%
%Title Page
\addtocounter{page}{-1}
\thispagestyle{empty}
\section{円運動と単振動}

\begin{enumerate}
\setcounter{enumi}{\thepage}
\small
\itemsep-4mm
\item ラジアンの定義\\
\item \vskip-1mm 角速度\\
\item \vskip-1mm 速度と角速度\\
\item \vskip-1mm 周期と角速度\\
\item \vskip-1mm 回転数と周期\\
\item \vskip-1mm 加速度と角速度\\
\item \vskip-1mm 単振動の変位 \\
\item \vskip-1mm 振動数と周期\\
\item \vskip-1mm 単振動の速度 \\
\item \vskip-1mm 単振動の加速度 \\
\item \vskip-1mm ばね振子の周期\\
\item \vskip-1mm 単振子の周期
\end{enumerate}
%%%%%%%%%%%%%%%%%%%%%%%%%%%%%



\newpage
%%%%%%%%%%%%%%%%%%%%%%%%%%%%%
\[
l = r \theta
\]


\vskip3mm
【ラジアンの定義】{\footnotesize (p. 118)}

\begin{tabular}{ccl}
$l$	&[m]	&:円弧の長さ\\
$r$	&[m]	&:半径\\
$\theta$	&[rad]	&:角度
\end{tabular}







\newpage
%%%%%%%%%%%%%%%%%%%%%%%%%%%%%
\[
\omega = \frac{\; \theta \;}{t}
\]


\vskip3mm
【角速度】{\footnotesize (p. 119)}

\begin{tabular}{ccl}
$\omega$	&[rad/s]	&:角速度\\
$\theta$	&[rad]	&:角度\\
$t$	&[s]	&:時間
\end{tabular}






\newpage
%%%%%%%%%%%%%%%%%%%%%%%%%%%%%
\[
v = \frac{\; l \;}{t} = r \omega
\]


\vskip3mm
【速度と角速度】{\footnotesize (p. 119)}

\begin{tabular}{ccl}
$v$	&[m/s]	&:速さ\\
$l$	&[m]	&:円弧の長さ\\
$t$	&[s]	&:時間\\
$r$	&[m]	&:半径\\
$\omega$	&[rad/s]	&:角速度\\
\end{tabular}




\newpage
%%%%%%%%%%%%%%%%%%%%%%%%%%%%%
\[
T = \frac{2\pi r}{v} = \frac{\; 2\pi \;}{\omega}
\]


\vskip3mm
【周期と角速度】{\footnotesize (p. 119)}

\begin{tabular}{ccl}
$T$	&[s]	&:周期\\
$r$	&[m]	&:半径\\
$v$	&[m/s]	&:速さ\\
$\omega$	&[rad/s]	&:角速度
\end{tabular}




\newpage
%%%%%%%%%%%%%%%%%%%%%%%%%%%%%
\[
n = \frac{1}{\; T \;}
\]


\vskip3mm
【回転数と周期】{\footnotesize (p. 120)}

\begin{tabular}{ccl}
$n$	& [Hz]	&:回転数\\
$T$	&[s]	&:周期
\end{tabular}





\newpage
%%%%%%%%%%%%%%%%%%%%%%%%%%%%%
\[
	a = r \omega^2
\]


\vskip3mm
【加速度と角速度】{\footnotesize (p. 122)}

\begin{tabular}{ccl}
$a$	&[m/s$^2$]	&:加速度\\
$r$	&[m]	&:半径\\
$\omega $	&[rad/s]	&:角速度
\end{tabular}





\newpage
%%%%%%%%%%%%%%%%%%%%%%%%%%%%%
\[
	x = A \sin (\omega t)
\]


\vskip3mm
【単振動の変位】{\footnotesize (p. 127)}

\begin{tabular}{ccl}
$x$	&[m]&	:変位\\
$A$	&[m]	&:振幅\\
$\omega $	&[rad/s]	&:角振動数 \\
$t$	&[s]	&:時刻 \\
\end{tabular}





\newpage
%%%%%%%%%%%%%%%%%%%%%%%%%%%%%
\[
f = \frac{1}{\; T \;}
\]


\vskip3mm
【振動数と周期】{\footnotesize (p. 128)}

\begin{tabular}{ccl}
$f$	& [Hz]	&:振動数\\
$T$	&[s]	&:周期
\end{tabular}




\newpage
%%%%%%%%%%%%%%%%%%%%%%%%%%%%%
\[
	v = A \omega \cos (\omega t)
\]


\vskip3mm
【単振動の速度】{\footnotesize (p. 129)}

\begin{tabular}{ccl}
$v$	&[m]&	:速度\\
$A$	&[m]	&:振幅\\
$\omega $	&[rad/s]	&:角振動数 \\
$t$	&[s]	&:時刻 \\
\end{tabular}



\newpage
%%%%%%%%%%%%%%%%%%%%%%%%%%%%%
\[
	a = -A \omega^2 \sin (\omega t)
\]


\vskip3mm
【単振動の加速度】{\footnotesize (p. 129)}

\begin{tabular}{ccl}
$a$	&[m]&	:加速度\\
$A$	&[m]	&:振幅\\
$\omega $	&[rad/s]	&:角振動数 \\
$t$	&[s]	&:時刻 \\
\end{tabular}



\newpage
%%%%%%%%%%%%%%%%%%%%%%%%%%%%%
\[
	T = 2\pi \sqrt{\frac{\; m \;}{k}}
\]


\vskip3mm
【ばね振子の周期】{\footnotesize (p. 132)}

\begin{tabular}{ccl}
$T$	&[s]&	:周期\\
$m$	&[kg]	&:質量\\
$k$	&[N/m]	&:ばね定数
\end{tabular}






\newpage
%%%%%%%%%%%%%%%%%%%%%%%%%%%%%
\[
	T = 2\pi \sqrt{\frac{l}{\; g \;}}
\]


\vskip3mm
【単振子の周期】{\footnotesize (p. 137)}

\begin{tabular}{ccl}
$T$	&[s]&	:周期\\
$l$	&[m]	&:振子の長さ\\
$g$	&[m/s$^2$]	&:重力加速度
\end{tabular}







\newpage
%%%%%%%%%%%%%%%%%%%%%%%%%%%%%
%Title Page
\addtocounter{page}{-1}
\thispagestyle{empty}
\section{万有引力の法則}

\begin{enumerate}
\setcounter{enumi}{\thepage}
\small
\itemsep-4mm
\item 万有引力の法則\\
\item 第一宇宙速度\\
\item 万有引力の位置エネルギー
\end{enumerate}
%%%%%%%%%%%%%%%%%%%%%%%%%%%%%




\newpage
%%%%%%%%%%%%%%%%%%%%%%%%%%%%%
\[
F = G \frac{m_1 m_2}{r^2}
\]


\vskip3mm
【万有引力の法則】{\footnotesize (p. 144)}

\begin{tabular}{ccl}
$F$	&[N]	&:力 \\
$G$	&{\small [N$\cdot \!\!$ m$^2 \!$/kg$^2$]}	&:万有引力定数 \\
$m_i$	&[kg]	&:質量 \\
$r$	&[m]	&:距離
\end{tabular}





\newpage
%%%%%%%%%%%%%%%%%%%%%%%%%%%%%
\[
v = \sqrt{\frac{G M}{R}}
\]


\vskip3mm
【第一宇宙速度】{\footnotesize (p. 148)}

\begin{tabular}{ccl}
$v$	&[m/s]	&:第一宇宙速度 \\
$G$	&{\small [N$\cdot \!\!$ m$^2 \!$/kg$^2$]}	&:万有引力定数 \\
$M$	&[kg]	&:地球の質量 \\
$R$	&[m]	&:地球の半径
\end{tabular}





\newpage
%%%%%%%%%%%%%%%%%%%%%%%%%%%%%
\[
U = - G \frac{M m}{r}
\]


\vskip3mm
【万有引力の位置エネルギー】\\
\hfill {\footnotesize (p. 152)}

\begin{tabular}{ccl}
$U$	&[J]	&:{\small 位置エネルギー} \\
$G$	&{\small [N$\cdot \!\!$ m$^2\!$/kg$^2$]}	&:万有引力定数 \\
$M$	&[kg]	&:質量 \\
$m$	&[kg]	&:質量 \\
$r$	&[m]	&:距離
\end{tabular}



\end{document}
